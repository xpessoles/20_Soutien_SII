\documentclass[10pt]{article}
\input{style/coursHeadings}
\input{style/programHeadings}
\input{style/macros_SII}
\input{style/macros_Titres}
\input{style/macros_Frames}

%Si le boolen xp est vrai : compilation pour xabi
%Sinon compilation Damien

\newif\ifprof
%\proftrue
\proffalse

\newif\ifxp
\xptrue
%\xpfalse

\newif\iftd
\tdtrue
%\tdfalse


\usepackage[%
    pdftitle={},
    pdfauthor={Xavier Pessoles},
    colorlinks=true,
    linkcolor=blue,
    citecolor=magenta]{hyperref}


\def\discipline{Sciences Industrielles de l'Ingénieur}
\def\xxtitre{%
\ifxp
3 \& 4 -- Étude Cinématique et Statique des systèmes de solides de la chaîne d'énergie
Analyse, Modélisation, Résolution
\else
\fi
}

\def\xxsoustitre{%
\ifxp
Chapitre 1 -- Notions de base 
\else
\fi}

\def\xxauteur{%
\ifxp
Xavier \textsc{Pessoles}
\else
\fi}

\def\xxpied{%
\ifxp
3\&4 : Cinématique et Statique\\
Ch. 1 : Notions de base -- Applications
\else
\fi}






%---------------------------------------------------------------------------


\begin{document}
\ifxp
\input{style/enteteXP}
\else
\input{style/enteteDI}
\fi


\subsection*{Exercice 1 -- Calculs de moments}

\begin{minipage}[c]{.68\linewidth}
\subparagraph*{}
\textit{Calculer $\vect{\mathcal{M}\left(A,\vect{F}\right)}$ en utilisant une méthode <<intuitive>> puis en utilisant la définition du moment.}
\end{minipage}\hfill
\begin{minipage}[c]{.3\linewidth}
\begin{center}
\includegraphics[width=\textwidth]{images/moment1}
\end{center}
\end{minipage}



\begin{minipage}[c]{.68\linewidth}
\subparagraph*{}
\textit{Calculer $\vect{\mathcal{M}\left(O,\vect{F}\right)}$ en utilisant une méthode <<intuitive>> puis en utilisant la définition du moment.}
\end{minipage}\hfill
\begin{minipage}[c]{.3\linewidth}
\begin{center}
\includegraphics[width=\textwidth]{images/moment5}
\end{center}
\end{minipage}

\begin{minipage}[c]{.68\linewidth}
\subparagraph*{}
\textit{Calculer $\vect{\mathcal{M}\left(O,\vect{F}\right)}$ en utilisant une méthode <<intuitive>> puis en utilisant la définition du moment.}

\end{minipage}\hfill
\begin{minipage}[c]{.3\linewidth}
\begin{center}
\includegraphics[width=\textwidth]{images/moment6}
\end{center}
\end{minipage}


\subsubsection*{Exercice 2 : Application du PFS}

\begin{minipage}[c]{.5\linewidth}
La balance romaine représentée ci-dessus est constituée d'un contrepoids 3 coulissant le long de la tige 2 (graduée).
\subparagraph*{}
\textit{Déterminer la masse de l'objet pesé sur le crochet 4.}

\subparagraph*{}
\textit{Déterminer la force extérieure sur 1 permettant de maintenir le système à l'équilibre.}

\end{minipage}\hfill
\begin{minipage}[c]{.48\linewidth}
\begin{center}
\includegraphics[width=\textwidth]{images/balance}
\end{center}
\end{minipage}



\begin{minipage}[c]{.5\linewidth}
Une échelle d’incendie (3), partiellement représentée figure ci-dessous, est articulée en A (pivot d'axe $(A,\vect{z})$) sur une tourelle (2). La tourelle peut pivoter (rotation d'axe $(D,\vect{y})$) par rapport au châssis du camion (1). Le levage est réalisé par un vérin hydraulique 4 + 5 (4 = tige, 5 = corps) articulé en B sur l'échelle et en C sur la tourelle, les liaisons en B et C sont des liaisons rotules de centres B et C.
L'étude est réalisée dans le plan de symétrie du dispositif, l'ensemble est en équilibre, la tourelle est à l'arrêt et le vérin est bloqué en position. ($\vect{P_3}$)($5\,000\;daN$) schématise le poids de l'échelle, le poids du vérin est négligé.

On note $\vect{AB}=x_1 \vect{x} + y_1 \vect{y}$ et $\vect{AC}=x_Z \vect{x} + y_Z \vect{y}$.

\subparagraph*{}
\textit{Déterminer les actions mécaniques en A, B et C. Les calculs seront faits sous forme littérale.}


\end{minipage}\hfill
\begin{minipage}[c]{.48\linewidth}
\begin{center}
\includegraphics[width=\textwidth]{images/echelle}
\end{center}
\end{minipage}


\vspace{.5cm}

La route est horizontale et toutes les actions exercées entre les roues et le sol sont considérées comme des ponctuelles. Le poids de la voiture est de 1 500 daN et celui de la remorque est de 800daN. 


On note $\vect{AG_1}=x_1 \vect{x} + y_1 \vect{y}$, 
$\vect{AB}=x_2 \vect{x}$, 
$\vect{AC}=x_3 \vect{x} + y_3 \vect{y}$,
$\vect{AG_2}=x_4 \vect{x} + y_4 \vect{y}$,
$\vect{AD}=x_5 \vect{x}$.

\subparagraph*{}
\textit{Déterminer les actions mécaniques en A, B C et D. Tous les calculs seront faits sous forme littérale.}

\begin{center}
\includegraphics[width=\textwidth]{images/remorque}
\end{center}

\subsection*{Exercice 3 -- Cinématique plane}

On donne :
\begin{itemize}
\item $\omega(3/0) = 1\, 500 tr/min$;
\item $CD = 5\, mm$;
\item échelle : 1cm pour 0,2 m/s.
\end{itemize}

\begin{minipage}[c]{.48\linewidth}
\begin{center}
\includegraphics[width=\textwidth]{images/schema}
\end{center}
\end{minipage} \hfill
\begin{minipage}[c]{.48\linewidth}
\begin{center}
\includegraphics[width=\textwidth]{images/schema}
\end{center}
\end{minipage}

\subparagraph*{Utilisation du CIR}

\textit{En justifiant, déterminer : 
\begin{enumerate}
\item $\vectv{C}{3}{0}$;
\item $\vectv{B}{2}{0}$;
\item $||\vectv{B}{2}{0}||$;
\item $||\vectv{E}{2}{0}||$.
\end{enumerate}}


\subparagraph*{Utilisation de l'équiprojectivité}


\textit{En justifiant, déterminer : 
\begin{enumerate}
\item $\vectv{C}{3}{0}$;
\item $\vectv{B}{2}{0}$;
\item $||\vectv{B}{2}{0}||$;
\item $||\vectv{E}{2}{0}||$.
\end{enumerate}}
\end{document}


